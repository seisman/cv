\section{Peer-reviewed Publications}
% AGU style: https://publications.agu.org/agu-grammar-and-style-guide/
\newcommand{\Revision}{\emph{under revision}}
\newcommand{\CS}{*} % corresponding author
\newcommand{\CF}{\textsuperscript{\#}} % co-first author

\CS corresponding author, \CF co-first author.
\begin{etaremune}
\item Li, J.\CS, \Me, Li, M., Chu, R. (2025).
	Small-Scale Heterogeneities in the Lowermost Mantle Near the Perm Anomaly.
	\emph{Journal of Geophysical Research: Solid Earth}, \emph{130}(8), e2025JB031160.
	\DOI{10.1029/2025JB031160}
\item Li, J.\CS, \Me\CS, Li, M., Sun, D., Mao Z., Dobrosavljevic V. (2025).
    Ultralow Velocity Zones at the Core-Mantle Boundary Near the Caroline Hotspot.
    \emph{Journal of Geophysical Research: Solid Earth}, \emph{130}(7), e2024JB030763.
    \DOI{10.1029/2024JB030763}
\item Li, J.\CS, Sun, D., \Me\ (2024).
    Localized Ultra-Low Velocity Zone as a Strong Scatterer at the Core-Mantle Boundary Beneath Central America.
    \emph{Journal of Geophysical Research: Solid Earth}, \emph{129}(12), e2024JB029287.
    \DOI{10.1029/2024JB029287}
\item Li, J.\CS, \Me, Sun, D., Tong, P. (2024).
    D'' structures beneath the East China Sea resolved by P-wave slowness anomalies.
    \emph{Journal of Geophysical Research: Solid Earth}, \emph{129}(11), e2024JB029584.
    \DOI{10.1029/2024JB029584}
\item \Me\ (2024).
    HinetPy: A Python package for accessing and processing NIED Hi-net seismic data.
    \emph{Journal of Open Source Software}, \emph{9}(98), 6840.
    \DOI{10.21105/joss.06840}
\item Li, J.\CS, Zhang, B., Sun, D., \Me, \JYao\ (2024).
    Detailed 3D structures of the western edge of the Pacific Large Low Velocity Province.
    \emph{Journal of Geophysical Research: Solid Earth}, \emph{129}(4), e2023JB028032.
    \DOI{10.1029/2023JB028032}
\item \Me\CS, \& \LWen\ (2023).
    Comment on ``Inner Core Rotation Captured by Earthquake Doublets and Twin Stations'' by Yang and Song.
    \emph{Geophysical Research Letters}, \emph{50}(15), e2023GL103173.
    \DOI{10.1029/2023GL103173}
\item \Me\CS, \SWei\CS, \WWang, \& \FWang\ (2022).
    Stress drops of intermediate-depth and deep earthquakes in the Tonga slab.
    \emph{Journal of Geophysical Research: Solid Earth}, \emph{127}, e2022JB025109.
    \DOI{10.1029/2022JB025109}
\item \JYao\CS, \Me, \LSun, \& \LWen\ (2021).
    Comment on ``Origin of temporal changes of inner-core seismic waves'' by Yang and Song (2020).
    \emph{Earth and Planetary Science Letters}, \emph{553}, 116640.
    \DOI{10.1016/j.epsl.2020.116640}
\item \SWei\CS, \PShearer, \CLithgowBertelloni, \LStixrude, \& \Me\ (2020).
    Oceanic plateau of the Hawaiian mantle plume head subducted to the uppermost lower mantle.
    \emph{Science}, \emph{370}, 983--987.
    \DOI{10.1126/science.abd0312}
\item \Me\CS, \MLv, \SWei, \SDorfman, \& \PShearer\ (2020).
    Global variations of Earth's 520- and 560-km discontinuities.
    \emph{Earth and Planetary Science Letters}, \emph{552}, 116600. \\
    \DOI{10.1016/j.epsl.2020.116600}
\item
    \PWessel\CS, \JLuis, \LUieda, \RScharroo, \FWobbe, \WSmith, \& \Me\ (2019).
    The Generic Mapping Tools Version 6.
    \emph{Geochemistry, Geophysics, Geosystems}, \emph{20}(11), 5556--5564.
    \DOI{10.1029/2019GC008515}
\item
    \JYao\CS, \Me, \LSun, \& \LWen\ (2019).
    Temporal change of seismic Earth's inner core phases: inner core differential rotation or temporal change of inner core surface?
    \emph{Journal of Geophysical Research: Solid Earth}, \emph{124}(7), 6720--6736.
    \DOI{10.1029/2019JB017532}
\item
    \WFan\CS, \SWei, \Me, \JMcGurie, \& \DWiens\ (2019).
    Complex and diverse rupture processes of the 2018 Mw 8.2 and Mw 7.9 Tonga-Fiji deep earthquakes.
    \emph{Geophysical Research Letters}, \emph{46}(5), 2434--2448.
    \DOI{10.1029/2018GL080997}
\item
    \JYao\CF\CS, \Me\CF, \ZLu, \LSun, \& \LWen\ (2018).
    Triggered seismicity after North Korea's 3 September 2017 nuclear test.
    \emph{Seismological Research Letters}, \emph{89}(6), 2085--2093.
    \DOI{10.1785/0220180135}
\item
    \JYao\CF\CS, \Me\CF, \LSun, \& \LWen\ (2018).
    Source characteristics of North Korea's 3 September 2017 nuclear test.
    \emph{Seismological Research Letters}, \emph{89}(6), 2078--2084.
    \DOI{10.1785/0220180134}
\item
    \Me\CF\CS, \JYao\CF, \& \LWen\ (2018).
    Collapse and earthquake swarm after North Korea's 3 September 2017 nuclear test.
    \emph{Geophysical Research Letters}, \emph{45}(9), 3976--3983.
    \DOI{10.1029/2018GL077649}
\item
    \LWen\CS, \Me, \& \JYao\ (2018).
    Seismic structure and dynamic process of the Earth's inner core and its boundary.
    \emph{Chinese Journal of Geophysics}, \emph{61}(3), 803--818.
    \DOI{10.6038/cjg2018L0500} [in Chinese]
\item
    \Me, \& \LWen\CS\ (2017).
    Seismological evidence for a localized mushy zone at the Earth's inner core boundary.
    \emph{Nature Communications}, 8, 165.
    \DOI{10.1038/s41467-017-00229-9}
\item
    \XChen\CS, \Me, \& \LWen\ (2015).
    Microseismic sources during Hurricane Sandy.
    \emph{Journal of Geophysical Research: Solid Earth}, \emph{120}(9), 6386--6403.
    \DOI{10.1002/2015JB012282}
\item \MZhang\CS, \Me, \& \LWen\ (2014).
    A new method for earthquake depth determination: stacking multiple-station autocorrelograms.
    \emph{Geophysical Journal International}, \emph{197}(2), 1107--1116.
    \DOI{10.1093/gji/ggu044}
\end{etaremune}

%\subsection*{Papers submitted/under revision}
%\begin{etaremune}
%\end{etaremune}
