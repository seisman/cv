%%%%%%%%%%%%%%%%%%%%%%%%%%%%%%%%%%%%%%%%%%%%%%%%%%%%%%%%%%%%%%%%%%%%%%%%%%%%%%%
% 学术简历 LaTeX 模板
%%%%%%%%%%%%%%%%%%%%%%%%%%%%%%%%%%%%%%%%%%%%%%%%%%%%%%%%%%%%%%%%%%%%%%%%%%%%%%%
\documentclass[11pt,a4paper]{article}

%%%%%%%%%%%%%%%%%%%%%%%%%%%%%%%%%%%%%%%%%%%%%%%%%%%%%%%%%%%%%%%%%%%%%%%%%%%%%%%
% 自定义信息
%%%%%%%%%%%%%%%%%%%%%%%%%%%%%%%%%%%%%%%%%%%%%%%%%%%%%%%%%%%%%%%%%%%%%%%%%%%%%%%
% 个人信息
\newcommand{\Title}{学术简历}
\newcommand{\Name}{田冬冬}
\newcommand{\Role}{特任教授}
\newcommand{\Email}{dtian@cug.edu.cn}
\newcommand{\Website}{me.seisman.info}
\newcommand{\Github}{seisman}
\newcommand{\Orcid}{0000-0001-7967-1197}
\newcommand{\Affiliation}{中国地质大学(武汉)\\ 地球物理与空间信息学院}
\newcommand{\Address}{湖北省武汉市洪山区鲁磨路 388 号\\ 基委楼 205 室}

% 在文章列表中引用作者和合作者
\newcommand{\Me}{\textbf{Tian, D.}} % 加粗显示
\newcommand{\XChen}{Chen, X.}
\newcommand{\SDorfman}{Dorfman, S. M.}
\newcommand{\WFan}{Fan, W.}
\newcommand{\CLithgowBertelloni}{Lithgow-Bertelloni, C.}
\newcommand{\ZLu}{Lu, Z.}
\newcommand{\JLuis}{Luis, J.}
\newcommand{\MLv}{Lv, M.}
\newcommand{\JMcGurie}{McGuire, J. J.}
\newcommand{\RScharroo}{Scharroo, R.}
\newcommand{\PShearer}{Shearer, P. M.}
\newcommand{\WSmith}{Smith, W. H. F.}
\newcommand{\LStixrude}{Stixrude, L.}
\newcommand{\LSun}{Sun, L.}
\newcommand{\LUieda}{Uieda, L.}
\newcommand{\WWang}{Wang, W.}
\newcommand{\FWang}{Wang, F.}
\newcommand{\SWei}{Wei, S. S.}
\newcommand{\LWen}{Wen, L.}
\newcommand{\PWessel}{Wessel, P.}
\newcommand{\DWiens}{Wiens, D. A.}
\newcommand{\FWobbe}{Wobbe, F.}
\newcommand{\JYao}{Yao, J.}
\newcommand{\MZhang}{Zhang, M.}
\newcommand{\YZhang}{Zhang, Y.}

% 一些命令
\newcommand{\DOI}[1]{doi:\href{https://dx.doi.org/#1}{#1}}

% 中文支持
\usepackage[fontset=fandol]{ctex}
% 设置页面边距
\usepackage[margin=2.5cm]{geometry}
% 设置字体
\usepackage{fontspec}
\usepackage[default,semibold]{sourcesanspro}

% Use fontawesome icons
\usepackage[fixed]{fontawesome5}

% 对列表各项逆序编号(用于对文章进行编号)
\usepackage[itemsep=2pt]{etaremune}

% 控制文字字号
\usepackage{anyfontsize}

% 以“年/月”格式显示日期
\usepackage{datetime}
\newdateformat{monthyear}{\THEYEAR/\THEMONTH}

% 设置每节标题的前后空白
\usepackage{titlesec}
\titlespacing*{\section}{0pt}{1ex}{1ex}
% 设置 section 不显示编号,且可生成目录
\titleformat{\section}{\normalfont\Large\bfseries}{}{0pt}{}

% 设置行间距
\renewcommand{\baselinestretch}{1.2}
% 设置表格的垂直距离
\renewcommand{\arraystretch}{1.2}
\setlength{\parindent}{0pt} % no indent for paragraph

% 设置列表中各项之间的间距
\usepackage{enumitem}
\setlist{itemsep=0pt}

% 长表格
\usepackage{tabularx}
\usepackage{ltablex}
\usepackage{environ}
\NewEnviron{EntriesTable}[3]{
\vspace{-1.25em}
\begin{tabularx}{\textwidth}{p{#1\textwidth}@{\hspace{#2\textwidth}}p{#3\textwidth}}
    \BODY
\end{tabularx}
}

% 获取总页数
\usepackage{lastpage}

% 设置页眉页脚
\usepackage{fancyhdr}
\pagestyle{fancy}
\fancyhf{}
\chead{
    \itshape
    \fontsize{10pt}{12pt}\selectfont
    \Name
    \hspace{0.2cm} -- \hspace{0.2cm}
    \Title
    \hspace{0.2cm} -- \hspace{0.2cm}
    \monthyear\today
}
\rhead{}
\cfoot{\fontsize{10pt}{0}\selectfont \thepage/\pageref*{LastPage}}
\renewcommand{\headrulewidth}{0pt}

% 使用自定义颜色
\usepackage[usenames,dvipsnames]{xcolor}

% PDF 元信息以及超链接
\usepackage[colorlinks=true]{hyperref}
\hypersetup{ % document metadata
    pdftitle = {\Name\ - \Title},
    pdfauthor = {\Name},
    linkcolor=black,
    citecolor=black,
    filecolor=black,
    urlcolor=MidnightBlue,
}
%%%%%%%%%%%%%%%%%%%%%%%%%%%%%%%%%%%%%%%%%%%%%%%%%%%%%%%%%%%%%%%%%%%%%%%%%%%%%%%

\begin{document}

% CV 封面页
\thispagestyle{empty} % 首页不显示页眉页脚
\begin{center}
\kaishu
{\fontsize{24pt}{0}\selectfont \Name \hspace{1ex}} \\[0.4cm]
{\fontsize{16pt}{0}\selectfont \Role} \\[0.2cm]
\end{center}
\begin{minipage}[t]{0.7\textwidth}
  \kaishu
  \fontsize{12pt}{15pt}\selectfont
  \Affiliation
  \\
  \Address
\end{minipage}
\begin{minipage}[t]{0.3\textwidth}
  \kaishu
  \fontsize{12pt}{15pt}\selectfont
  \begin{flushleft}
    \faEnvelope \href{mailto:\Email}{\texttt{\Email}}
	\\
	\faOrcid \href{https://orcid.org/\Orcid}{\Orcid}
	\\
    \faGlobe \href{https://\Website}{\Website}
	\\
	\faGithub \href{https://github.com/\Github}{\Github}
  \end{flushleft}
\end{minipage}
\vspace{0.2cm}

\section*{Education}

\begin{tabular}{p{0.18\textwidth} p{0.82\textwidth}}
2012 -- 2018 & Ph.D in Geophysics, University of Science and Technology of China, Hefei, China \\
2008 -- 2012 & B.S. in Geophysics, University of Science and Technology of China, Hefei, China
\end{tabular}

\section*{工作经历}

\begin{tabular}{p{0.18\textwidth} p{0.82\textwidth}}
2021/11 至今 & 特任教授,中国地质大学(武汉),中国湖北省武汉市 \\
2018/08--2021/09 & 博士后研究助理,密西根州立大学,美国密西根州东兰辛 \\
\end{tabular}

\section*{研究方向及兴趣}

\begin{itemize}
\item 地球深部结构
\item 地震震源理论及观测
\item 地震波传播理论
\end{itemize}

\section*{Professional Societies \& Activities}

\begin{tabular}{ll}
2012 -- present & Member of the American Geophysical Union (AGU) \\
2016 -- present & Technical support of \href{http://chinageorefmodel.org/}{China Seismological Reference Model} \\
2016 -- present & Founder and primary contributor of \href{http://gmt-china.org/}{GMT Chinese Community} \\
2017 -- present & Peer-reviewer of scientific journals: \textit{Geophysical Research Letters} (1) \\
\end{tabular}

\section*{Awards \& Honors}

\begin{tabular}{p{0.05\textwidth} p{0.95\textwidth}}
2017 & \textbf{National Scholarship for Doctoral Students}, Ministry of Education, China \newline
        (\textit{Awarded to top 5\% of all doctoral students in China}) \\
2014 & Guanghua Scholarship for Graduate Students, Guanghua Education Fund, China \\
\end{tabular}

\section{Received Funds}

\begin{itemize}
\item Startup, CUG One Hundred Talents Program, \textyen\ 2,000k (2021--2026)
\end{itemize}

% AGU style: https://publications.agu.org/agu-grammar-and-style-guide/
\newcommand{\Revision}{\textit{under revision}}
\newcommand{\CS}{*} % corresponding author
\newcommand{\CF}{\textsuperscript{\#}} % co-first author

\section*{Peer-reviewed Publications}
\CS corresponding author, \CF co-first author.

\begin{etaremune}
\item
    Yao, J., \Tian, Sun, L., \& Wen, L.
    Temporal change of seismic Earth's inner core phases: inner core differential rotation or temporal change of inner core surface?
    \textit{Journal of Geophysical Research: Solid Earth},
    \DOI{10.1029/2019JB017532}.
    \textit{in press}.
\item
    Fan, W., S.S. Wei, \Tian, J.J. McGuire, and D.A. Wiens. (2019).
    Complex and diverse rupture processes of the 2018 Mw 8.2 and Mw 7.9 Tonga-Fiji deep earthquakes.
    \textit{Geophysical Research Letters}, \textit{46}(5), 2434--2448.
    \DOI{10.1029/2018GL080997}
\item
    Yao, J., \Tian\CF, Lu, Z., Sun, L., \& Wen, L. (2018).
    Triggered seismicity after North Korea's 3 September 2017 nuclear test.
    \textit{Seismological Research Letters}, \textit{89}(6), 2085--2093.
    \DOI{10.1785/0220180135}
\item
    Yao, J., \Tian\CF, Sun, L., \& Wen, L. (2018).
	Source characteristics of North Korea's 3 September 2017 nuclear test.
    \textit{Seismological Research Letters}, \textit{89}(6), 2078--2084.
    \DOI{10.1785/0220180134}
\item
    \Tian, Yao, J., \& Wen, L. (2018).
    Collapse and earthquake swarm after North Korea's 3 September 2017 nuclear test.
    \textit{Geophysical Research Letters}, \textit{45}(9), 3976--3983.
    \DOI{10.1029/2018GL077649}
\item
    Wen, L., \Tian, \& Yao, J. (2018).
    Seismic structure and dynamic process of the Earth's inner core and its boundary.
    \textit{Chinese Journal of Geophysics}, \textit{61}(3), 803--818.
    \DOI{10.6038/cjg2018L0500} [in Chinese]
\item
    \Tian, \& Wen, L. (2017).
    Seismological evidence for a localized mushy zone at the Earth's inner core boundary.
    \textit{Nature communications}, 8, 165.
    \DOI{10.1038/s41467-017-00229-9}
\item
    Chen, X., \Tian, \& Wen, L. (2015).
    Microseismic sources during hurricane sandy.
    \textit{Journal of Geophysical Research: Solid Earth}, \textit{120}(9), 6386--6403.
    \DOI{10.1002/2015JB012282}
\item Zhang, M., \Tian, \& Wen, L. (2014).
    A new method for earthquake depth determination: stacking multiple-station autocorrelograms.
    \textit{Geophysical Journal International}, \textit{197}(2), 1107--1116.\\
    \DOI{10.1093/gji/ggu044}
\end{etaremune}

\subsection*{Papers submitted/under revision}
\begin{etaremune}
\item
    Wessel, P., Luis, J., Uieda, L., Scharroo, R., Wobbe, F., Smith, W. H. F., \& \Tian,
    The Generic Mapping Tools, Version 6.
    \textit{submitted to Geochemistry, Geophysics, Geosystems}.
\end{etaremune}

\subsection*{Papers in Preparation}
\begin{etaremune}
\item
    \Tian, \& Wen, L.
    Improved relative moment tensor inversion method and applications to clusters of small earthquakes.
\item
    \Tian, \& Wen, L.
    Three types of Earth's inner core boundary.
\item
    \Tian, \& Wen, L.
    Simulating wave propagation in a faulted medium using a 3D finite difference method.
\end{etaremune}

\section*{Meeting Abstracts}
\begin{etaremune}
\item \Tian, Wang, W. \& Wei, S. S. (2019)
	Source spectra and stress drop of deep earthquakes in the Tonga subduction zone.
	Abstract S13C-0458 presented at 2019 AGU Fall Meeting, San Francisco, CA, USA.
\item
    \Tian, Wei, S. S., \& Shearer, M. P. (2019)
    Global variations of the 520-km discontinuity.
    Presented at Gordon Research Conference: Interior of the Earth, South Hadley, MA, USA.
\item
    \Tian, Wei, S. S., \& Shearer, M. P. (2018)
    Global variations of the 520-km discontinuity.
    Abstract DI31C-0024 presented at 2018 AGU Fall Meeting, Washington, DC, USA.
\item
    \Tian, Yao, J., \& Wen, L. (2017).
    Collapse and earthquake swarm after North Korea's 3 September 2017 nuclear test.
    Abstract S43H-2968 presented at 2017 AGU Fall Meeting, New Orleans, LA, USA.
\item
    \Tian, \& Wen, L. (2017).
    Three types of Earth's inner core boundary.
    Abstract DI33B-0404 presented at 2017 AGU Fall Meeting, New Orleans, LA, USA.
\item
    Yao, J., \Tian, \& Wen, L. (2017).
    High-precision location, yield and tectonic release of North Korea's 3 September 2017 nuclear test.
    Abstract S43H-2967 presented at 2017 AGU Fall Meeting, New Orleans, LA, USA.
\item
    Yao, J., \Tian, Sun, L., \& Wen, L. (2017).
    Temporal change of seismic Earth's inner core phases: Inner core differential rotation or temporal change of inner core surface?
    Abstract DI33B-0405 presented at 2017 AGU Fall Meeting, New Orleans, LA, USA.
\item
    \Tian, \& Wen, L. (2017).
    Seismological evidence for a localized mushy zone at the Earth's inner core boundary.
    Presented at Gordon Research Conference: Interior of the Earth, South Hadley, MA, USA.
\item
    Yao, J., \Tian, Sun, L., \& Wen, L. (2017).
    Temporal change of seismic Earth's inner core phases: Inner core differential rotation or temporal change of inner core surface?
    Presented at Gordon Research Conference: Interior of the Earth, South Hadley, MA, USA.
\item
    \Tian, \& Wen, L. (2016).
    Seismic structures of the Earth's inner core boundary beneath the Bearing sea and Mexico.
    Abstract DI43A-2657 presented at 2016 AGU Fall Meeting, San Francisco, CA, USA.
\item
    \Tian, \& Wen, L. (2015).
    Varying seismic property of the Earth's inner core boundary.
    Abstract DI33A-2606 presented at 2015 AGU Fall Meeting, San Francisco, CA, USA.
\item
    \Tian, \& Wen, L. (2014).
    Seismic study on the properties of the Earth's inner core boundary.
    Abstract DI31B-4269 presented at 2014 AGU Fall Meeting, San Francisco, CA, USA.
\item
    Chen, X., \Tian, \& Wen, L. (2013).
    Seismic tracking of hurricane sandy.
    Abstract S11A-2296 presented at 2013 AGU Fall Meeting, San Francisco, CA, USA.
\item
    \Tian, \& Wen, L. (2013).
    Regional topography variation of Earth's inner core boundary.
    Abstract DI23A-2282 presented at 2013 AGU Fall Meeting, San Francisco, CA, USA.
\item
    Zhang, M., \Tian, \& Wen, L. (2013).
    A new method for earthquake determination: stacking multiple-station autocorrelograms.
    Abstract S51A-2301 presented at 2013 AGU Fall Meeting, San Francisco, CA, USA.
\item
    \Tian, \& Wen, L. (2012).
    Simulating wave propagation in a faulted medium using a 3D finite difference method.
    Abstract S43A-2458 presented at 2012 AGU Fall Meeting, San Francisco, CA, USA.
\end{etaremune}

\section*{Talks}
\begin{etaremune}
\item
	Global variations of Earth's 520- and 560-km discontinuities,
	Nanjing University,
	Jan. 7, 2021.
\item
	Global variations of Earth's 520- and 560-km discontinuities,
	Department of Earth and Space Sciences, Southern University of Science and Techbology,
	Nov. 27, 2020.
	\invited.
\item
    Global variations of the 520-km discontinuity.
    \textit{ 2nd Annual Earth and Environmental Sciences Student Research Symposium},
    Department of Earth and Environmental Sciences, Michigan State University, East Lansing, MI, USA.
    Feb. 23, 2019.
    \textbf{[5 minutes lightning talk]}
\item
    Collapse and earthquake swarm after North Korea's 2017 nuclear test.
    \textit{Institute of Geology and Geophysics, Chinese Academy of Sciences}, Beijing, China.
    Jun. 15, 2018.
\item
    Seismological evidence for a localized mushy zone at the Earth's inner core boundary.
    \textit{Institute of Geology and Geophysics, Chinese Academy of Sciences}, Beijing, China.
    Jun. 15, 2018.
    \invited
\item
    Fine-scale structure of the Earth's inner core boundary and aftershocks of North Korea's 2017 nuclear test.
    \textit{Institute of Earthquake Forcasting, China Earthquake Administration}, Beijing, China.
    Jun. 14, 2018.
\item
    Seismological evidence for a localized mushy zone at the Earth's inner core boundary.
    \textit{2017 Annual Meeting of Chinese Geoscience Union (CGU)}, Beijing, China.
    Oct. 17, 2017.
    \invited
\item
    Getting started with GMT in 60 minutes.
    \textit{Workshop on Analysis and Applications of Crustal Deformation Data}, Wuhan, China.
    Sep. 21, 2016.
    \invited
\item
    Seismic study on the properties of the Earth's inner core boundary.
    \textit{China Earthquake Networks Center}, Beijing, China.
    Jun. 30, 2016.
    \invited
\end{etaremune}

\section{野外经历}

\begin{itemize}
\item \textbf{LEEP} (\textbf{L}ake \textbf{E}rie \textbf{E}arthquake ex\textbf{P}eriment),
      2018/10/12--2018/10/16,在 Erie 湖周边安装 8 个宽频带地震仪
\end{itemize}

\section{开源软件}

\begin{EntriesTable}{0.10}{0.02}{0.88}
2014至今 & \textbf{HinetPy} | \url{https://github.com/seisman/HinetPy/} \newline
           用于从 Hi-net 网站申请和处理地震波形数据的 Python 包 \newline
           唯一开发者 \\
2018至今 & \textbf{PyGMT} | \url{https://www.pygmt.org/} \newline
           地学制图工具 GMT 的 Python 接口 \newline
           核心开发者 \\
\end{EntriesTable}


\end{document}
